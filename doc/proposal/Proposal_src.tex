% --- START OF FILE proposal.tex ---

% Ensure 10pt font size is specified
\documentclass[conference, 10pt]{IEEEtran} 

% --- Preamble (Packages, etc.) ---
\usepackage{cite}
\usepackage{amsmath,amssymb,amsfonts}
\usepackage{algorithmic}
\usepackage{graphicx} % Needed for including figures
\usepackage{textcomp}
\usepackage{xcolor}
\usepackage{url} % Needed for \url{} 

% Optional: For code listings (choose one or use verbatim)
% \usepackage{listings} 
% \lstset{
%   basicstyle=\footnotesize\ttfamily,
%   breaklines=true,
% }

\def\BibTeX{{\rm B\kern-.05em{\sc i\kern-.025em b}\kern-.08em
    T\kern-.1667em\lower.7ex\hbox{E}\kern-.125emX}}
    
\begin{document}

% --- Title Block ---
\title{Accelerating Convolution Kernels on Multi-Core CPUs using OpenMP Parallelization}
% --- AUTHORS ---
\author{
    \IEEEauthorblockN{1\textsuperscript{st} [Student One Name]}
    \IEEEauthorblockA{\textit{Computer and Systems Eng. Dept.} \\
    \textit{Faculty of Engineering, Ain Shams University}\\
    Cairo, Egypt \\
    Student1ID@eng.asu.edu.eg}
\and
    \IEEEauthorblockN{2\textsuperscript{nd} [Mohamed Salah Fathy]}
    \IEEEauthorblockA{\textit{Computer and Systems Eng. Dept.} \\
    \textit{Faculty of Engineering, Ain Shams University}\\
    Cairo, Egypt \\
    21p0117@eng.asu.edu.eg}
\and
    \IEEEauthorblockN{3\textsuperscript{rd} [Salma Mohamed Youssef]}
    \IEEEauthorblockA{\textit{Computer and Systems Eng. Dept.} \\
    \textit{Faculty of Engineering, Ain Shams University}\\
    Cairo, Egypt \\
    21p0148@eng.asu.edu.eg}
\and
    \IEEEauthorblockN{4\textsuperscript{th} [Salma Hisham Hassan Wagdy]}
    \IEEEauthorblockA{\textit{Computer and Systems Eng. Dept.} \\
    \textit{Faculty of Engineering, Ain Shams University}\\
    Cairo, Egypt \\
    21p0124@eng.asu.edu.eg}
}


\maketitle 

% --- Abstract & Keywords ---
\begin{abstract}
Convolution is a fundamental and computationally intensive operation widely used in computer vision and image processing. This project investigates the performance limitations of sequential convolution implementations and explores acceleration using multi-core CPU architectures. We employ profiling techniques to identify bottlenecks, apply OpenMP directives to parallelize the core convolution loops, and conduct a systematic evaluation of the resulting speedup and scalability. The objective is to demonstrate significant performance improvements achievable through standard parallel programming techniques for this common computational kernel, contextualized within the principles of automatic parallelization analysis.
\end{abstract}

\begin{IEEEkeywords}
Parallel Computing, OpenMP, Image Convolution, Image Processing, Performance Analysis, Multi-core CPU, Kernel Acceleration, Speedup, Scalability
\end{IEEEkeywords}

% ====================================================================
% --- MAIN PROPOSAL CONTENT STARTS HERE ---
% ====================================================================

% --- Introduction ---
\section{Introduction}
\textit{{\color{blue} % Optional color for visibility
[Convolution is a foundational operation in image processing and computer vision, essential for tasks such as edge detection, blurring, and feature extraction. And more recently, It is also a core component in many deep learning architectures. Despite its ubiquity, convolution is known to be computationally expensive, especially when applied to large images or multiple channels. When executed sequentially on modern multi-core systems, convolution operations often underutilize the available hardware resources, leading to suboptimal performance and increased processing time.

As multi-core processors have become the standard in contemporary computing environments, the need to exploit their capabilities through parallelization has grown significantly. Automatic parallelization techniques aim to bridge this gap by transforming sequential code into parallel code without requiring extensive manual intervention. However, the parallelization of convolution operations poses several challenges, including handling data dependencies, managing memory access patterns, and minimizing synchronization overhead \cite{hager2021hpc}. These challenges are particularly critical in image processing, where the convolution kernel accesses neighboring pixel values, making naive parallelization error-prone.

To address these issues, researchers have explored parallelization frameworks such as OpenMP, which provides a simple yet powerful interface for shared-memory parallelism. When applied thoughtfully, OpenMP directives can yield substantial performance improvements in convolution tasks \cite{farber2011openmp}. Prior work has demonstrated that even modest parallelization of convolution loops can lead to significant gains in execution time and scalability \cite{toth2016convolution}.

In this project, we focus on analyzing and parallelizing a sequential image convolution kernel using OpenMP on a multi-core CPU. Our approach involves profiling the code to identify performance bottlenecks, applying parallel constructs to the core convolution routines, and evaluating the resulting speedup. Through this study, we aim to provide undergraduate students with hands-on experience in automatic parallelization and highlight the practical benefits of parallel programming in optimizing a widely-used computational kernel.]
}}

% --- PROBLEM DEFINITION ---
\section{Problem Definition}
\textit{{\color{blue} % Optional color for visibility
[Placeholder for Problem Definition: This section will precisely define the scope of the project. It will present the illustrative sequential image convolution kernel (e.g., Fig.~\ref{fig:seq_code}), explain its relevance, connect it to dependency analysis concepts (referencing \cite{idkfactchecking2025} where applicable), and clearly state the specific research problem and tasks: profile, analyze dependencies, parallelize using OpenMP, and evaluate. This placeholder text should be replaced with the actual content, including code figures.]
}}

% Example Figure (referenced above, keep or adapt)
\begin{figure}[htbp]
\begin{verbatim}
// Placeholder for sequential_convolve code
void sequential_convolve(...) {
  // ... loops ...
}
\end{verbatim}
\caption{Illustrative Sequential Convolution Kernel.}
\label{fig:seq_code}
\end{figure}


% --- EVALUATION PLAN ---
\section{Proposed Evaluation}
\textit{{\color{blue} % Optional color for visibility
[Placeholder for Proposed Evaluation: This section will detail the methodology for evaluating the parallelized convolution kernel. It includes establishing the sequential baseline, verifying correctness, the benchmarking procedure (hardware, software environment, varying thread counts), key performance metrics (Execution Time, Speedup, Efficiency), planned benchmark inputs (images, kernels), and the criteria for success. This placeholder text should be replaced with the specific evaluation plan.
\cite{hbdscientific2025}]
}}





% ====================================================================
% --- END OF MAIN PROPOSAL CONTENT ---
% ====================================================================


\section*{Acknowledgment} 
% Add acknowledgments
We thank Dr. Islam Tharwat Abdel Halim and Eng. Hassan Ahmed for their guidance.


% --- Bibliography Setup (Keep these lines) ---
\bibliographystyle{IEEEtran} 
\bibliography{references} % file is references.bib

\end{document}
% --- END OF FILE proposal.tex ---