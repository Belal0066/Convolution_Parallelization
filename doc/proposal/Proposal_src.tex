% --- START OF FILE proposal.tex ---

% Ensure 10pt font size is specified
\documentclass[conference, 10pt]{IEEEtran} 

% --- Preamble (Packages, etc.) ---
\usepackage{cite}
\usepackage{amsmath,amssymb,amsfonts}
\usepackage{algorithmic}
\usepackage{graphicx} % Needed for including figures
\usepackage{textcomp}
\usepackage{xcolor}
\usepackage{url} % Needed for \url{} 

% Optional: For code listings (choose one or use verbatim)
% \usepackage{listings} 
% \lstset{
%   basicstyle=\footnotesize\ttfamily,
%   breaklines=true,
% }

\def\BibTeX{{\rm B\kern-.05em{\sc i\kern-.025em b}\kern-.08em
    T\kern-.1667em\lower.7ex\hbox{E}\kern-.125emX}}
    
\begin{document}

% --- Title Block ---
\title{Accelerating Convolution Kernels on Multi-Core CPUs using OpenMP Parallelization}
% --- AUTHORS ---
\author{
    \IEEEauthorblockN{1\textsuperscript{st} [Student One Name]}
    \IEEEauthorblockA{\textit{Computer and Systems Eng. Dept.} \\
    \textit{Faculty of Engineering, Ain Shams University}\\
    Cairo, Egypt \\
    Student1ID@eng.asu.edu.eg}
\and
    \IEEEauthorblockN{2\textsuperscript{nd} [Mohamed Salah Fathy]}
    \IEEEauthorblockA{\textit{Computer and Systems Eng. Dept.} \\
    \textit{Faculty of Engineering, Ain Shams University}\\
    Cairo, Egypt \\
    21p0117@eng.asu.edu.eg}
\and
    \IEEEauthorblockN{3\textsuperscript{rd} [Salma Mohamed Youssef]}
    \IEEEauthorblockA{\textit{Computer and Systems Eng. Dept.} \\
    \textit{Faculty of Engineering, Ain Shams University}\\
    Cairo, Egypt \\
    21p0148@eng.asu.edu.eg}
\and
    \IEEEauthorblockN{4\textsuperscript{th} [Salma Hisham Hassan Wagdy]}
    \IEEEauthorblockA{\textit{Computer and Systems Eng. Dept.} \\
    \textit{Faculty of Engineering, Ain Shams University}\\
    Cairo, Egypt \\
    21p0124@eng.asu.edu.eg}
}


\maketitle 

% --- Abstract & Keywords ---
\begin{abstract}
Convolution is a fundamental and computationally intensive operation widely used in computer vision and image processing. This project investigates the performance limitations of sequential convolution implementations and explores acceleration using multi-core CPU architectures. We employ profiling techniques to identify bottlenecks, apply OpenMP directives to parallelize the core convolution loops, and conduct a systematic evaluation of the resulting speedup and scalability. The objective is to demonstrate significant performance improvements achievable through standard parallel programming techniques for this common computational kernel, contextualized within the principles of automatic parallelization analysis.
\end{abstract}

\begin{IEEEkeywords}
Parallel Computing, OpenMP, Image Convolution, Image Processing, Performance Analysis, Multi-core CPU, Kernel Acceleration, Speedup, Scalability
\end{IEEEkeywords}

% ====================================================================
% --- MAIN PROPOSAL CONTENT STARTS HERE ---
% ====================================================================

% --- Introduction ---
\section{Introduction}
\textit{{\color{blue} % Optional color for visibility
[Convolution is a foundational operation in image processing and computer vision, widely used for tasks such as edge detection, blurring, and feature extraction. It's also a core component in many deep learning architectures. 
Despite its ubiquity, convolution is known to be computationally intensive, particularly when applied to large images or multi-channel data. 
Sequential implementations of convolution often underutilize the modern multi-core CPU architectures, leading to suboptimal performance and increased processing time.
Automatic parallelization offers a promising approach to address these performance limitations by transforming sequential code into parallel code with minimal programmer intervention 
This process involves analyzing the code to identify opportunities for parallelism, dividing computations into parallel tasks, and coordinating their execution to maximize hardware utilization. 
However, automatic parallelization presents challenges such as managing data dependencies, balancing workload distribution, and minimizing communication overhead, especially for operations like convolution that rely on local neighborhoods in the input data.
Message Passing Interface (MPI) has emerged as a standard and powerful framework for distributed memory parallel programming, enabling efficient communication and task coordination across multiple processes. 
While MPI offers fine-grained control over data distribution and communication, effectively applying it to convolution operations requires careful handling of data partitioning, boundary conditions, and inter-process communication \cite{hager2021hpc}.
Convolution is particularly challenging to parallelize due to its data dependencies — each output value depends on a local neighborhood in the input matrix. 
Naively distributing data without overlap leads to incorrect results at boundaries. 
Typically, parallelizing convolution with MPI could involve for example strategies such as halo exchange or ghost cell communication to preserve correctness \cite{toth2016convolution}. 
Previous research has demonstrated that MPI-based parallelization of convolution can lead to substantial performance improvements, especially when scaling across multiple compute nodes \cite{farber2011openmp}.
Through this work, we aim to analyze the performance limitations of a sequential convolution kernel and implement a parallel version using MPI. We will begin by profiling the original code to identify bottlenecks, 
then apply domain decomposition techniques to parallelize the convolution routine across multiple processes. The parallel implementation will be benchmarked to evaluate speedup and scalability, 
providing insight into the practical benefits and challenges of automatic parallelization in distributed systems.
The project underscores the importance of mastering parallel programming paradigms like MPI to efficiently solve computational problems in real-world image processing tasks.]
}}

% --- PROBLEM DEFINITION ---
\section{Problem Definition}
\textit{{\color{blue} % Optional color for visibility
[Placeholder for Problem Definition: This section will precisely define the scope of the project. It will present the illustrative sequential image convolution kernel (e.g., Fig.~\ref{fig:seq_code}), explain its relevance, connect it to dependency analysis concepts (referencing \cite{idkfactchecking2025} where applicable), and clearly state the specific research problem and tasks: profile, analyze dependencies, parallelize using OpenMP, and evaluate. This placeholder text should be replaced with the actual content, including code figures.]
}}

% Example Figure (referenced above, keep or adapt)
\begin{figure}[htbp]
\begin{verbatim}
// Placeholder for sequential_convolve code
void sequential_convolve(...) {
  // ... loops ...
}
\end{verbatim}
\caption{Illustrative Sequential Convolution Kernel.}
\label{fig:seq_code}
\end{figure}


% --- EVALUATION PLAN ---
\section{Proposed Evaluation}
\textit{{\color{blue} % Optional color for visibility
[Placeholder for Proposed Evaluation: This section will detail the methodology for evaluating the parallelized convolution kernel. It includes establishing the sequential baseline, verifying correctness, the benchmarking procedure (hardware, software environment, varying thread counts), key performance metrics (Execution Time, Speedup, Efficiency), planned benchmark inputs (images, kernels), and the criteria for success. This placeholder text should be replaced with the specific evaluation plan.
\cite{hbdscientific2025}]
}}





% ====================================================================
% --- END OF MAIN PROPOSAL CONTENT ---
% ====================================================================


\section*{Acknowledgment} 
% Add acknowledgments
We thank Dr. Islam Tharwat Abdel Halim and Eng. Hassan Ahmed for their guidance.


% --- Bibliography Setup (Keep these lines) ---
\bibliographystyle{IEEEtran} 
\bibliography{references} % file is references.bib

\end{document}
% --- END OF FILE proposal.tex ---